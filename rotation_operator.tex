\section{The rotation operator}\sidenote{\texttt{rotation\textunderscore operator.tex}}
We  can represent $\ket{+z}$ as a vector in a Cartesian coordinate
system. If we rotate this vector anticlockwise by 90\degree, we get
$\ket{+x}$. Mathematically, this can be written as,

\begin{equation}
\ket{+x} = \hat R \left(\frac{\pi}{2} \mathbf{j} \right) \ket{+z},
\label{eq: rotation_ket}
\end{equation}

where $\hat R$ is called the \emph{rotation operator}. The terms inside the
brackets indicate that the ket is rotated by an angle 90\degree about the $y-$
axis \sidenote{Try to be aware of this: we are using the Cartisian coordinate
system to represent the spin vectors visually. These spin vectors live in a
complex vector space, and are different from the vectors we use to represent
classical angular momentum. The Cartesian coordinate system, vector, and the
complex vector space are just some of the tools we have developed to reconstruct
the physical world in our minds. They do \emph{not} exist in nature, but they
are the building blocks of our mental models of the world. We conduct
experiments to check if these mental models are indeed in tune with the real
world out there. This activity of building mental models and checking if they
correctly mimic the reality is called \emph{science}.}. The same rotation operator
rotates $\ket{-z}$ into $\ket{-x}$, $\ket{-y}$ into $\ket{-x}$, and so on.

Let's now see the effect of this rotation operator on the most general spin-$1/2$
state given by \cref{eq: general_state}.

\begin{equation}
	\hat R \left(\frac{\pi}{2} \mathbf{j} \right) \ket{\psi} = 
	c_{+} \hat R \left(\frac{\pi}{2} \mathbf{j} \right) \ket{+z}
       + c_{-} \hat R \left(\frac{\pi}{2} \mathbf{j} \right) \ket{-z}
	\label{eq: rotation_generalstate}
\end{equation}

Using \cref{eq: rotation_ket} in \cref{eq: rotation_generalstate}, we get,

\begin{equation}
\hat R \left(\frac{\pi}{2} \mathbf{j} \right) \ket{\psi} = 
	c_{+} \ket{+x} + c_{-} \ket{-x}.
\end{equation}

Yes, the rotation operator has changed the basis in which the state $\ket{\psi}$
was desribed!

\subsection{The adjoint operator}
In the previous section section, we discussed the effect of a rotation operator
on a ket vector. What if we are dealing with bra vectors, instead of kets? In 
other words, what is the bra equation corresponding to \cref{eq: rotation_ket}?
Is it $\bra{+x} = \bra{+z} \hat R \left(\frac{\pi}{2} \mathbf{j} \right)$? Let's
check.

We know that the amplitude for a state to be in itself is 1. Mathematically, this
is written as,

\begin{equation}
\langle +x | +x \rangle = 1 \nonumber
\end{equation}

Assuming that $\bra{+x} = \bra{+z} \hat R \left(\frac{\pi}{2} \mathbf{j} \right)$ is
correct, we can write,

\begin{equation}
\bra{+z} \hat R \left(\frac{\pi}{2} \mathbf{j} \right) 
\hat R \left(\frac{\pi}{2} \mathbf{j} \right)\ket{+z} = 1
\end{equation}

We have, $\hat R \left(\frac{\pi}{2} \mathbf{j} \right) \hat R \left(\frac{\pi}{2} \mathbf{j} \right) = \ket{-z}$.
This leads to,

\begin{equation}
\langle +z | -z \rangle = 1
\end{equation}

This is an absurd result because, the LHS is equal to zero. Therefore, our assumption that
$\bra{+x} = \bra{+z} \hat R \left(\frac{\pi}{2} \mathbf{j} \right)$ is incorrect. Then, what
is the bra equation corresponding to \cref{eq: rotation_ket}?

Let's now introduce a new operator $\hat R^{\dag}$, called the \emph{adjoint} of $\hat R$.
Now the bra equation becomes,

\begin{equation}
	\bra{+x} = \bra{+z} \hat R^{\dag} \left(\frac{\pi}{2} \mathbf{j} \right) 
\end{equation}

Normalisation condition is now satisfied:

\begin{equation*}
\begin{aligned}
\langle+x \mid+x\rangle &=\left\langle+z\left|\hat{R}^{\dagger}\left(\frac{\pi}{2} \mathbf{j}\right) \hat{R}\left(\frac{\pi}{2} \mathbf{j}\right)\right|+z\right\rangle \\
%
&=\left\langle+z\left|\hat{R}^{\dagger}\left(\frac{\pi}{2} \mathbf{j}\right)\right|+x \right \rangle \\
&=\langle+z \mid+z\rangle \\
&=1
\end{aligned}
\end{equation*}

Therefore, in order for the ket to satisfy normalisation condition even after
rotation, it is necessary to introduce a new operator whose effect is to
negate the action of the rotation operator. Mathematically, the new operator
is represented by the adjoint of the rotation operator. 
Naturally, successive action of
rotation operator and its adjoint will leave the ket unchanged. It is useful
to interpret this to be equivalent to the action of a new type of operator
called \emph{identity operator}. Mathematically, this can be written as,

\begin{equation}
	{R} \hat{R}^{\dagger} =\hat{I}
\end{equation}

In summary, a rotation operator $\hat{R}\left(\theta \mathbf{n}\right)$ changes
a ket by rotating it by an angle $\theta$ around the axis specified by the unit
vector $\mathbf{n}$.

\subsection{The generator of rotations}
The rotation of a vector by an infinitesimally small angle $d\phi$ about the $z-$ axis
can be presented by the operator:
\begin{equation}
\hat{R}(d\phi \mathbf{k}) =\hat{I}-\frac{i}{\hbar} \hat{J}_{z} d \phi
\end{equation}

Here, $\hat{J}_{z}$ is called the \emph{generator of rotations}. We can rotate a 
vector by a finite angle $\phi$ by repeatedly carrying out inifinitesimally
small rotations of magnitude $d\, \phi$. \sidenote{In Eq 14, since the LHS has the dimensions 
of angle, the RHS should also have the same
dimensions. That means, the dimensions of $\hbar$ and $\hat J_{z}$ should
cancel. Since, $\hbar$ has the dimensions of angular momentum, $\hat J_{z}$
should also have the dimensions of angular momentum (otherwise they can't
cancel each other).}

The adjoint of the above rotation operator is obtained by: 
\begin{enumerate}
	\item Replacing the complex numbers by their respective conjugates
	\item Replacing the generator of rotations operator by its adjoint
\end{enumerate}

Therefore,
\begin{equation}
\hat {R}^{\dagger}(d \phi \mathbf{ k)}=\hat{I}+\frac{i}{\hbar} \hat{J}_{z}^{\dagger} d \phi
\end{equation}

Since, $\hat{R} \hat{R}^{\dagger} =\hat{I}$, we have,

\begin{equation*}
\begin{aligned}
\left(\hat{I}-\frac{i}{\hbar} \hat{J}_{z} d \phi\right)\left(\hat{I}+\frac{i}{\hbar} \hat{J}_{z}^{\dagger} d \phi\right)=\hat{I}\\
%
\hat{I}+\frac{i}{\hbar} \hat{J}_{z}^{\dagger} d \phi-\frac{i}{\hbar} \hat{J}_{z} d \phi+\frac{1}{\hbar^{2}} \hat{J}_{z} \hat{J}_{z}^{\dagger}(d \phi)^{2}= \hat{I}\\
\end{aligned}
\end{equation*}

Since the angle $d\phi$ is infinitesimally small, the last term on the LHS 
can be neglected. This leads to,
\begin{equation}
\hat{I}+\frac{i}{\hbar}\left(\hat{J}_{z}^{\dagger}-\hat{J}_{z}\right) d \phi=\hat{I},
\end{equation}

which can be true only if $\hat{J}_{z}^{\dagger}=\hat{J}_{z}$. Operators that are
equal to their own adjoints are said to be \emph{self-adjoint} or \emph{Hermitian}.
\sidenote{The generaot of rotations is a Hermitian operator, but the rotation operator is not.}

There is one more way to discover that $\hat J_{z}$ is Hermitian. If we rotate 
the ket in clockwise direction, the rotation operator becomes:

\begin{equation}
\begin{aligned}
\hat{R}(- \phi \mathbf{k}) & = \hat{I}-\frac{i}{\hbar} \hat{J}_{z} (- d \phi) \\
	& = \hat{I}+\frac{i}{\hbar} \hat{J}_{z} d \phi
\end{aligned}
\end{equation}

By the definition of adjoint of rotation operator, we have,
\begin{equation}
\begin{aligned}
\hat{R}(- d \phi \mathbf{k}) &= \hat{R}^{\dagger}(d \phi \mathbf{k}) \\
	\hat{I}-\frac{i}{\hbar} \hat{J}_{z} (- d \phi) &= \hat{I}+\frac{i}{\hbar} \hat{J}_{z}^{\dagger} d \phi
\end{aligned}
\end{equation}

This can be true only if $\hat{J}_{z} = \hat{J}_{z}^{\dagger}$.
