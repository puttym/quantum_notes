\section{Identity operators}
\sidenote{identity\textunderscore operator.tex}
The general state $\ket{\psi}$ of a spin-$\frac{1}{2}$ particle is:
\begin{equation*}
\ket{\psi} =c_{+} \ket{+z} + c_{-}\ket{-z}
\end{equation*}
Multiplying by by $\ket{+z}$, we get,
\begin{equation*}
\braket{+z | \psi} = c_{+} \braket{+z | +z} 
\end{equation*}
or,
\begin{equation*}
c_{+} = \braket{+z | \psi}
\end{equation*}

Similarly, by multiplying by by $\ket{-z}$, we get,
\begin{equation*}
c_{-} = \braket{-z | \psi}
\end{equation*}

Now, $\ket{\psi}$ can be written as,
\begin{align*}
\ket{\psi} &= \braket{+z | \psi} \ket{+z} + \braket{-z | \psi} \ket{-z}  \\
%
&= \ket{+z} \braket{+z | \psi} + \ket{-z} \braket{-z | \psi} \\
%
&= \ket{+z} \bra{+z} \ket{\psi} + \ket{+z} \bra{-z} \ket{\psi}
\end{align*}

Or,
\begin{equation*}
\boxed{
\ket{\psi} = \left[ \ket{+z} \bra{+z} + \ket{-z} \bra{-z} \right]\, \ket{\psi}
}
\end{equation*}

On the right hand side, the term in the square bracket can be considered as an 
operator that acts on the state $\ket{\psi}$, but doesn't change it at all. It is 
example of an \emph{identity operator}.
